\section{process}\label{process}

The \texttt{process} object is a global object and can be accessed from
anywhere. It is an instance of
\href{events.html\#events_class_events_eventemitter}{EventEmitter}.

\subsection{Exit Codes}\label{exit-codes}

Node will normally exit with a \texttt{0} status code when no more async
operations are pending. The following status codes are used in other
cases:

\begin{itemize}
\itemsep1pt\parskip0pt\parsep0pt
\item
  \texttt{1} \textbf{Uncaught Fatal Exception} - There was an uncaught
  exception, and it was not handled by a domain or an
  \texttt{uncaughtException} event handler.
\item
  \texttt{2} - Unused (reserved by Bash for builtin misuse)
\item
  \texttt{3} \textbf{Internal JavaScript Parse Error} - The JavaScript
  source code internal in Node's bootstrapping process caused a parse
  error. This is extremely rare, and generally can only happen during
  development of Node itself.
\item
  \texttt{4} \textbf{Internal JavaScript Evaluation Failure} - The
  JavaScript source code internal in Node's bootstrapping process failed
  to return a function value when evaluated. This is extremely rare, and
  generally can only happen during development of Node itself.
\item
  \texttt{5} \textbf{Fatal Error} - There was a fatal unrecoverable
  error in V8. Typically a message will be printed to stderr with the
  prefix \texttt{FATAL   ERROR}.
\item
  \texttt{6} \textbf{Non-function Internal Exception Handler} - There
  was an uncaught exception, but the internal fatal exception handler
  function was somehow set to a non-function, and could not be called.
\item
  \texttt{7} \textbf{Internal Exception Handler Run-Time Failure} -
  There was an uncaught exception, and the internal fatal exception
  handler function itself threw an error while attempting to handle it.
  This can happen, for example, if a
  \texttt{process.on('uncaughtException')} or
  \texttt{domain.on('error')} handler throws an error.
\item
  \texttt{8} - Unused. In previous versions of Node, exit code 8
  sometimes indicated an uncaught exception.
\item
  \texttt{9} - \textbf{Invalid Argument} - Either an unknown option was
  specified, or an option requiring a value was provided without a
  value.
\item
  \texttt{10} \textbf{Internal JavaScript Run-Time Failure} - The
  JavaScript source code internal in Node's bootstrapping process threw
  an error when the bootstrapping function was called. This is extremely
  rare, and generally can only happen during development of Node itself.
\item
  \texttt{12} \textbf{Invalid Debug Argument} - The \texttt{-{}-debug}
  and/or \texttt{-{}-debug-brk} options were set, but an invalid port
  number was chosen.
\item
  \texttt{\textgreater{}128} \textbf{Signal Exits} - If Node receives a
  fatal signal such as \texttt{SIGKILL} or \texttt{SIGHUP}, then its
  exit code will be \texttt{128} plus the value of the signal code. This
  is a standard Unix practice, since exit codes are defined to be 7-bit
  integers, and signal exits set the high-order bit, and then contain
  the value of the signal code.
\end{itemize}

\subsection{Event: `exit'}\label{event-exit}

Emitted when the process is about to exit. There is no way to prevent
the exiting of the event loop at this point, and once all \texttt{exit}
listeners have finished running the process will exit. Therefore you
\textbf{must} only perform \textbf{synchronous} operations in this
handler. This is a good hook to perform checks on the module's state
(like for unit tests). The callback takes one argument, the code the
process is exiting with.

Example of listening for \texttt{exit}:

\begin{Shaded}
\begin{Highlighting}[]
\OtherTok{process}\NormalTok{.}\FunctionTok{on}\NormalTok{(}\StringTok{'exit'}\NormalTok{, }\KeywordTok{function}\NormalTok{(code) \{}
  \CommentTok{// do *NOT* do this}
  \FunctionTok{setTimeout}\NormalTok{(}\KeywordTok{function}\NormalTok{() \{}
    \OtherTok{console}\NormalTok{.}\FunctionTok{log}\NormalTok{(}\StringTok{'This will not run'}\NormalTok{);}
  \NormalTok{\}, }\DecValTok{0}\NormalTok{);}
  \OtherTok{console}\NormalTok{.}\FunctionTok{log}\NormalTok{(}\StringTok{'About to exit with code:'}\NormalTok{, code);}
\NormalTok{\});}
\end{Highlighting}
\end{Shaded}

\subsection{Event: `uncaughtException'}\label{event-uncaughtexception}

Emitted when an exception bubbles all the way back to the event loop. If
a listener is added for this exception, the default action (which is to
print a stack trace and exit) will not occur.

Example of listening for \texttt{uncaughtException}:

\begin{Shaded}
\begin{Highlighting}[]
\OtherTok{process}\NormalTok{.}\FunctionTok{on}\NormalTok{(}\StringTok{'uncaughtException'}\NormalTok{, }\KeywordTok{function}\NormalTok{(err) \{}
  \OtherTok{console}\NormalTok{.}\FunctionTok{log}\NormalTok{(}\StringTok{'Caught exception: '} \NormalTok{+ err);}
\NormalTok{\});}

\FunctionTok{setTimeout}\NormalTok{(}\KeywordTok{function}\NormalTok{() \{}
  \OtherTok{console}\NormalTok{.}\FunctionTok{log}\NormalTok{(}\StringTok{'This will still run.'}\NormalTok{);}
\NormalTok{\}, }\DecValTok{500}\NormalTok{);}

\CommentTok{// Intentionally cause an exception, but don't catch it.}
\FunctionTok{nonexistentFunc}\NormalTok{();}
\OtherTok{console}\NormalTok{.}\FunctionTok{log}\NormalTok{(}\StringTok{'This will not run.'}\NormalTok{);}
\end{Highlighting}
\end{Shaded}

Note that \texttt{uncaughtException} is a very crude mechanism for
exception handling.

Don't use it, use \href{domain.html}{domains} instead. If you do use it,
restart your application after every unhandled exception!

Do \emph{not} use it as the node.js equivalent of
\texttt{On Error Resume Next}. An unhandled exception means your
application - and by extension node.js itself - is in an undefined
state. Blindly resuming means \emph{anything} could happen.

Think of resuming as pulling the power cord when you are upgrading your
system. Nine out of ten times nothing happens - but the 10th time, your
system is bust.

You have been warned.

\subsection{Signal Events}\label{signal-events}

Emitted when the processes receives a signal. See sigaction(2) for a
list of standard POSIX signal names such as SIGINT, SIGHUP, etc.

Example of listening for \texttt{SIGINT}:

\begin{Shaded}
\begin{Highlighting}[]
\CommentTok{// Start reading from stdin so we don't exit.}
\OtherTok{process}\NormalTok{.}\OtherTok{stdin}\NormalTok{.}\FunctionTok{resume}\NormalTok{();}

\OtherTok{process}\NormalTok{.}\FunctionTok{on}\NormalTok{(}\StringTok{'SIGINT'}\NormalTok{, }\KeywordTok{function}\NormalTok{() \{}
  \OtherTok{console}\NormalTok{.}\FunctionTok{log}\NormalTok{(}\StringTok{'Got SIGINT.  Press Control-D to exit.'}\NormalTok{);}
\NormalTok{\});}
\end{Highlighting}
\end{Shaded}

An easy way to send the \texttt{SIGINT} signal is with
\texttt{Control-C} in most terminal programs.

Note:

\begin{itemize}
\itemsep1pt\parskip0pt\parsep0pt
\item
  \texttt{SIGUSR1} is reserved by node.js to start the debugger. It's
  possible to install a listener but that won't stop the debugger from
  starting.
\item
  \texttt{SIGTERM} and \texttt{SIGINT} have default handlers on
  non-Windows platforms that resets the terminal mode before exiting
  with code \texttt{128 + signal number}. If one of these signals has a
  listener installed, its default behaviour will be removed (node will
  no longer exit).
\item
  \texttt{SIGPIPE} is ignored by default, it can have a listener
  installed.
\item
  \texttt{SIGHUP} is generated on Windows when the console window is
  closed, and on other platforms under various similar conditions, see
  signal(7). It can have a listener installed, however node will be
  unconditionally terminated by Windows about 10 seconds later. On
  non-Windows platforms, the default behaviour of \texttt{SIGHUP} is to
  terminate node, but once a listener has been installed its default
  behaviour will be removed.
\item
  \texttt{SIGTERM} is not supported on Windows, it can be listened on.
\item
  \texttt{SIGINT} from the terminal is supported on all platforms, and
  can usually be generated with \texttt{CTRL+C} (though this may be
  configurable). It is not generated when terminal raw mode is enabled.
\item
  \texttt{SIGBREAK} is delivered on Windows when \texttt{CTRL+BREAK} is
  pressed, on non-Windows platforms it can be listened on, but there is
  no way to send or generate it.
\item
  \texttt{SIGWINCH} is delivered when the console has been resized. On
  Windows, this will only happen on write to the console when the cursor
  is being moved, or when a readable tty is used in raw mode.
\item
  \texttt{SIGKILL} cannot have a listener installed, it will
  unconditionally terminate node on all platforms.
\item
  \texttt{SIGSTOP} cannot have a listener installed.
\end{itemize}

Note that Windows does not support sending Signals, but node offers some
emulation with \texttt{process.kill()}, and
\texttt{child\_process.kill()}: - Sending signal \texttt{0} can be used
to search for the existence of a process - Sending \texttt{SIGINT},
\texttt{SIGTERM}, and \texttt{SIGKILL} cause the unconditional exit of
the target process.

\subsection{process.stdout}\label{process.stdout}

A \texttt{Writable Stream} to \texttt{stdout}.

Example: the definition of \texttt{console.log}

\begin{Shaded}
\begin{Highlighting}[]
\OtherTok{console}\NormalTok{.}\FunctionTok{log} \NormalTok{= }\KeywordTok{function}\NormalTok{(d) \{}
  \OtherTok{process}\NormalTok{.}\OtherTok{stdout}\NormalTok{.}\FunctionTok{write}\NormalTok{(d + }\StringTok{'}\CharTok{\textbackslash{}n}\StringTok{'}\NormalTok{);}
\NormalTok{\};}
\end{Highlighting}
\end{Shaded}

\texttt{process.stderr} and \texttt{process.stdout} are unlike other
streams in Node in that writes to them are usually blocking. They are
blocking in the case that they refer to regular files or TTY file
descriptors. In the case they refer to pipes, they are non-blocking like
other streams.

To check if Node is being run in a TTY context, read the \texttt{isTTY}
property on \texttt{process.stderr}, \texttt{process.stdout}, or
\texttt{process.stdin}:

\begin{Shaded}
\begin{Highlighting}[]
\NormalTok{$ node -p }\StringTok{"Boolean(process.stdin.isTTY)"}
\KeywordTok{true}
\NormalTok{$ echo }\StringTok{"foo"} \NormalTok{| node -p }\StringTok{"Boolean(process.stdin.isTTY)"}
\KeywordTok{false}

\NormalTok{$ node -p }\StringTok{"Boolean(process.stdout.isTTY)"}
\KeywordTok{true}
\NormalTok{$ node -p }\StringTok{"Boolean(process.stdout.isTTY)"} \NormalTok{| cat}
\KeywordTok{false}
\end{Highlighting}
\end{Shaded}

See \href{tty.html\#tty_tty}{the tty docs} for more information.

\subsection{process.stderr}\label{process.stderr}

A writable stream to stderr.

\texttt{process.stderr} and \texttt{process.stdout} are unlike other
streams in Node in that writes to them are usually blocking. They are
blocking in the case that they refer to regular files or TTY file
descriptors. In the case they refer to pipes, they are non-blocking like
other streams.

\subsection{process.stdin}\label{process.stdin}

A \texttt{Readable Stream} for stdin. The stdin stream is paused by
default, so one must call \texttt{process.stdin.resume()} to read from
it.

Example of opening standard input and listening for both events:

\begin{Shaded}
\begin{Highlighting}[]
\OtherTok{process}\NormalTok{.}\OtherTok{stdin}\NormalTok{.}\FunctionTok{resume}\NormalTok{();}
\OtherTok{process}\NormalTok{.}\OtherTok{stdin}\NormalTok{.}\FunctionTok{setEncoding}\NormalTok{(}\StringTok{'utf8'}\NormalTok{);}

\OtherTok{process}\NormalTok{.}\OtherTok{stdin}\NormalTok{.}\FunctionTok{on}\NormalTok{(}\StringTok{'data'}\NormalTok{, }\KeywordTok{function}\NormalTok{(chunk) \{}
  \OtherTok{process}\NormalTok{.}\OtherTok{stdout}\NormalTok{.}\FunctionTok{write}\NormalTok{(}\StringTok{'data: '} \NormalTok{+ chunk);}
\NormalTok{\});}

\OtherTok{process}\NormalTok{.}\OtherTok{stdin}\NormalTok{.}\FunctionTok{on}\NormalTok{(}\StringTok{'end'}\NormalTok{, }\KeywordTok{function}\NormalTok{() \{}
  \OtherTok{process}\NormalTok{.}\OtherTok{stdout}\NormalTok{.}\FunctionTok{write}\NormalTok{(}\StringTok{'end'}\NormalTok{);}
\NormalTok{\});}
\end{Highlighting}
\end{Shaded}

\subsection{process.argv}\label{process.argv}

An array containing the command line arguments. The first element will
be `node', the second element will be the name of the JavaScript file.
The next elements will be any additional command line arguments.

\begin{Shaded}
\begin{Highlighting}[]
\CommentTok{// print process.argv}
\OtherTok{process}\NormalTok{.}\OtherTok{argv}\NormalTok{.}\FunctionTok{forEach}\NormalTok{(}\KeywordTok{function}\NormalTok{(val, index, array) \{}
  \OtherTok{console}\NormalTok{.}\FunctionTok{log}\NormalTok{(index + }\StringTok{': '} \NormalTok{+ val);}
\NormalTok{\});}
\end{Highlighting}
\end{Shaded}

This will generate:

\begin{Shaded}
\begin{Highlighting}[]
\NormalTok{$ node process}\FloatTok{-2.}\NormalTok{js one two=three four}
\DecValTok{0}\NormalTok{: node}
\DecValTok{1}\NormalTok{: }\OtherTok{/Users/mjr/work/node/process-2.js}
\OtherTok{2: one}
\OtherTok{3: two=three}
\OtherTok{4: four}
\end{Highlighting}
\end{Shaded}

\subsection{process.execPath}\label{process.execpath}

This is the absolute pathname of the executable that started the
process.

Example:

\begin{Shaded}
\begin{Highlighting}[]
\OtherTok{/usr/local/bin/node}
\end{Highlighting}
\end{Shaded}

\subsection{process.execArgv}\label{process.execargv}

This is the set of node-specific command line options from the
executable that started the process. These options do not show up in
\texttt{process.argv}, and do not include the node executable, the name
of the script, or any options following the script name. These options
are useful in order to spawn child processes with the same execution
environment as the parent.

Example:

\begin{Shaded}
\begin{Highlighting}[]
\NormalTok{$ node --harmony }\OtherTok{script}\NormalTok{.}\FunctionTok{js} \NormalTok{--version}
\end{Highlighting}
\end{Shaded}

results in process.execArgv:

\begin{Shaded}
\begin{Highlighting}[]
\NormalTok{[}\StringTok{'--harmony'}\NormalTok{]}
\end{Highlighting}
\end{Shaded}

and process.argv:

\begin{Shaded}
\begin{Highlighting}[]
\NormalTok{[}\StringTok{'/usr/local/bin/node'}\NormalTok{, }\StringTok{'script.js'}\NormalTok{, }\StringTok{'--version'}\NormalTok{]}
\end{Highlighting}
\end{Shaded}

\subsection{process.abort()}\label{process.abort}

This causes node to emit an abort. This will cause node to exit and
generate a core file.

\subsection{process.chdir(directory)}\label{process.chdirdirectory}

Changes the current working directory of the process or throws an
exception if that fails.

\begin{Shaded}
\begin{Highlighting}[]
\OtherTok{console}\NormalTok{.}\FunctionTok{log}\NormalTok{(}\StringTok{'Starting directory: '} \NormalTok{+ }\OtherTok{process}\NormalTok{.}\FunctionTok{cwd}\NormalTok{());}
\KeywordTok{try} \NormalTok{\{}
  \OtherTok{process}\NormalTok{.}\FunctionTok{chdir}\NormalTok{(}\StringTok{'/tmp'}\NormalTok{);}
  \OtherTok{console}\NormalTok{.}\FunctionTok{log}\NormalTok{(}\StringTok{'New directory: '} \NormalTok{+ }\OtherTok{process}\NormalTok{.}\FunctionTok{cwd}\NormalTok{());}
\NormalTok{\}}
\KeywordTok{catch} \NormalTok{(err) \{}
  \OtherTok{console}\NormalTok{.}\FunctionTok{log}\NormalTok{(}\StringTok{'chdir: '} \NormalTok{+ err);}
\NormalTok{\}}
\end{Highlighting}
\end{Shaded}

\subsection{process.cwd()}\label{process.cwd}

Returns the current working directory of the process.

\begin{Shaded}
\begin{Highlighting}[]
\OtherTok{console}\NormalTok{.}\FunctionTok{log}\NormalTok{(}\StringTok{'Current directory: '} \NormalTok{+ }\OtherTok{process}\NormalTok{.}\FunctionTok{cwd}\NormalTok{());}
\end{Highlighting}
\end{Shaded}

\subsection{process.env}\label{process.env}

An object containing the user environment. See environ(7).

\subsection{process.exit({[}code{]})}\label{process.exitcode}

Ends the process with the specified \texttt{code}. If omitted, exit uses
the `success' code \texttt{0}.

To exit with a `failure' code:

\begin{Shaded}
\begin{Highlighting}[]
\OtherTok{process}\NormalTok{.}\FunctionTok{exit}\NormalTok{(}\DecValTok{1}\NormalTok{);}
\end{Highlighting}
\end{Shaded}

The shell that executed node should see the exit code as 1.

\subsection{process.exitCode}\label{process.exitcode-1}

A number which will be the process exit code, when the process either
exits gracefully, or is exited via \texttt{process.exit()} without
specifying a code.

Specifying a code to \texttt{process.exit(code)} will override any
previous setting of \texttt{process.exitCode}.

\subsection{process.getgid()}\label{process.getgid}

Note: this function is only available on POSIX platforms (i.e.~not
Windows, Android)

Gets the group identity of the process. (See getgid(2).) This is the
numerical group id, not the group name.

\begin{Shaded}
\begin{Highlighting}[]
\KeywordTok{if} \NormalTok{(}\OtherTok{process}\NormalTok{.}\FunctionTok{getgid}\NormalTok{) \{}
  \OtherTok{console}\NormalTok{.}\FunctionTok{log}\NormalTok{(}\StringTok{'Current gid: '} \NormalTok{+ }\OtherTok{process}\NormalTok{.}\FunctionTok{getgid}\NormalTok{());}
\NormalTok{\}}
\end{Highlighting}
\end{Shaded}

\subsection{process.setgid(id)}\label{process.setgidid}

Note: this function is only available on POSIX platforms (i.e.~not
Windows, Android)

Sets the group identity of the process. (See setgid(2).) This accepts
either a numerical ID or a groupname string. If a groupname is
specified, this method blocks while resolving it to a numerical ID.

\begin{Shaded}
\begin{Highlighting}[]
\KeywordTok{if} \NormalTok{(}\OtherTok{process}\NormalTok{.}\FunctionTok{getgid} \NormalTok{&& }\OtherTok{process}\NormalTok{.}\FunctionTok{setgid}\NormalTok{) \{}
  \OtherTok{console}\NormalTok{.}\FunctionTok{log}\NormalTok{(}\StringTok{'Current gid: '} \NormalTok{+ }\OtherTok{process}\NormalTok{.}\FunctionTok{getgid}\NormalTok{());}
  \KeywordTok{try} \NormalTok{\{}
    \OtherTok{process}\NormalTok{.}\FunctionTok{setgid}\NormalTok{(}\DecValTok{501}\NormalTok{);}
    \OtherTok{console}\NormalTok{.}\FunctionTok{log}\NormalTok{(}\StringTok{'New gid: '} \NormalTok{+ }\OtherTok{process}\NormalTok{.}\FunctionTok{getgid}\NormalTok{());}
  \NormalTok{\}}
  \KeywordTok{catch} \NormalTok{(err) \{}
    \OtherTok{console}\NormalTok{.}\FunctionTok{log}\NormalTok{(}\StringTok{'Failed to set gid: '} \NormalTok{+ err);}
  \NormalTok{\}}
\NormalTok{\}}
\end{Highlighting}
\end{Shaded}

\subsection{process.getuid()}\label{process.getuid}

Note: this function is only available on POSIX platforms (i.e.~not
Windows, Android)

Gets the user identity of the process. (See getuid(2).) This is the
numerical userid, not the username.

\begin{Shaded}
\begin{Highlighting}[]
\KeywordTok{if} \NormalTok{(}\OtherTok{process}\NormalTok{.}\FunctionTok{getuid}\NormalTok{) \{}
  \OtherTok{console}\NormalTok{.}\FunctionTok{log}\NormalTok{(}\StringTok{'Current uid: '} \NormalTok{+ }\OtherTok{process}\NormalTok{.}\FunctionTok{getuid}\NormalTok{());}
\NormalTok{\}}
\end{Highlighting}
\end{Shaded}

\subsection{process.setuid(id)}\label{process.setuidid}

Note: this function is only available on POSIX platforms (i.e.~not
Windows, Android)

Sets the user identity of the process. (See setuid(2).) This accepts
either a numerical ID or a username string. If a username is specified,
this method blocks while resolving it to a numerical ID.

\begin{Shaded}
\begin{Highlighting}[]
\KeywordTok{if} \NormalTok{(}\OtherTok{process}\NormalTok{.}\FunctionTok{getuid} \NormalTok{&& }\OtherTok{process}\NormalTok{.}\FunctionTok{setuid}\NormalTok{) \{}
  \OtherTok{console}\NormalTok{.}\FunctionTok{log}\NormalTok{(}\StringTok{'Current uid: '} \NormalTok{+ }\OtherTok{process}\NormalTok{.}\FunctionTok{getuid}\NormalTok{());}
  \KeywordTok{try} \NormalTok{\{}
    \OtherTok{process}\NormalTok{.}\FunctionTok{setuid}\NormalTok{(}\DecValTok{501}\NormalTok{);}
    \OtherTok{console}\NormalTok{.}\FunctionTok{log}\NormalTok{(}\StringTok{'New uid: '} \NormalTok{+ }\OtherTok{process}\NormalTok{.}\FunctionTok{getuid}\NormalTok{());}
  \NormalTok{\}}
  \KeywordTok{catch} \NormalTok{(err) \{}
    \OtherTok{console}\NormalTok{.}\FunctionTok{log}\NormalTok{(}\StringTok{'Failed to set uid: '} \NormalTok{+ err);}
  \NormalTok{\}}
\NormalTok{\}}
\end{Highlighting}
\end{Shaded}

\subsection{process.getgroups()}\label{process.getgroups}

Note: this function is only available on POSIX platforms (i.e.~not
Windows, Android)

Returns an array with the supplementary group IDs. POSIX leaves it
unspecified if the effective group ID is included but node.js ensures it
always is.

\subsection{process.setgroups(groups)}\label{process.setgroupsgroups}

Note: this function is only available on POSIX platforms (i.e.~not
Windows, Android)

Sets the supplementary group IDs. This is a privileged operation,
meaning you need to be root or have the CAP\_SETGID capability.

The list can contain group IDs, group names or both.

\subsection{process.initgroups(user,
extra\_group)}\label{process.initgroupsuser-extraux5fgroup}

Note: this function is only available on POSIX platforms (i.e.~not
Windows, Android)

Reads /etc/group and initializes the group access list, using all groups
of which the user is a member. This is a privileged operation, meaning
you need to be root or have the CAP\_SETGID capability.

\texttt{user} is a user name or user ID. \texttt{extra\_group} is a
group name or group ID.

Some care needs to be taken when dropping privileges. Example:

\begin{Shaded}
\begin{Highlighting}[]
\OtherTok{console}\NormalTok{.}\FunctionTok{log}\NormalTok{(}\OtherTok{process}\NormalTok{.}\FunctionTok{getgroups}\NormalTok{());         }\CommentTok{// [ 0 ]}
\OtherTok{process}\NormalTok{.}\FunctionTok{initgroups}\NormalTok{(}\StringTok{'bnoordhuis'}\NormalTok{, }\DecValTok{1000}\NormalTok{);   }\CommentTok{// switch user}
\OtherTok{console}\NormalTok{.}\FunctionTok{log}\NormalTok{(}\OtherTok{process}\NormalTok{.}\FunctionTok{getgroups}\NormalTok{());         }\CommentTok{// [ 27, 30, 46, 1000, 0 ]}
\OtherTok{process}\NormalTok{.}\FunctionTok{setgid}\NormalTok{(}\DecValTok{1000}\NormalTok{);                     }\CommentTok{// drop root gid}
\OtherTok{console}\NormalTok{.}\FunctionTok{log}\NormalTok{(}\OtherTok{process}\NormalTok{.}\FunctionTok{getgroups}\NormalTok{());         }\CommentTok{// [ 27, 30, 46, 1000 ]}
\end{Highlighting}
\end{Shaded}

\subsection{process.version}\label{process.version}

A compiled-in property that exposes \texttt{NODE\_VERSION}.

\begin{Shaded}
\begin{Highlighting}[]
\OtherTok{console}\NormalTok{.}\FunctionTok{log}\NormalTok{(}\StringTok{'Version: '} \NormalTok{+ }\OtherTok{process}\NormalTok{.}\FunctionTok{version}\NormalTok{);}
\end{Highlighting}
\end{Shaded}

\subsection{process.versions}\label{process.versions}

A property exposing version strings of node and its dependencies.

\begin{Shaded}
\begin{Highlighting}[]
\OtherTok{console}\NormalTok{.}\FunctionTok{log}\NormalTok{(}\OtherTok{process}\NormalTok{.}\FunctionTok{versions}\NormalTok{);}
\end{Highlighting}
\end{Shaded}

Will print something like:

\begin{Shaded}
\begin{Highlighting}[]
\NormalTok{\{ }\DataTypeTok{http_parser}\NormalTok{: }\StringTok{'1.0'}\NormalTok{,}
  \DataTypeTok{node}\NormalTok{: }\StringTok{'0.10.4'}\NormalTok{,}
  \DataTypeTok{v8}\NormalTok{: }\StringTok{'3.14.5.8'}\NormalTok{,}
  \DataTypeTok{ares}\NormalTok{: }\StringTok{'1.9.0-DEV'}\NormalTok{,}
  \DataTypeTok{uv}\NormalTok{: }\StringTok{'0.10.3'}\NormalTok{,}
  \DataTypeTok{zlib}\NormalTok{: }\StringTok{'1.2.3'}\NormalTok{,}
  \DataTypeTok{modules}\NormalTok{: }\StringTok{'11'}\NormalTok{,}
  \DataTypeTok{openssl}\NormalTok{: }\StringTok{'1.0.1e'} \NormalTok{\}}
\end{Highlighting}
\end{Shaded}

\subsection{process.config}\label{process.config}

An Object containing the JavaScript representation of the configure
options that were used to compile the current node executable. This is
the same as the ``config.gypi'' file that was produced when running the
\texttt{./configure} script.

An example of the possible output looks like:

\begin{Shaded}
\begin{Highlighting}[]
\NormalTok{\{ }\DataTypeTok{target_defaults}\NormalTok{:}
   \NormalTok{\{ }\DataTypeTok{cflags}\NormalTok{: [],}
     \DataTypeTok{default_configuration}\NormalTok{: }\StringTok{'Release'}\NormalTok{,}
     \DataTypeTok{defines}\NormalTok{: [],}
     \DataTypeTok{include_dirs}\NormalTok{: [],}
     \DataTypeTok{libraries}\NormalTok{: [] \},}
  \DataTypeTok{variables}\NormalTok{:}
   \NormalTok{\{ }\DataTypeTok{host_arch}\NormalTok{: }\StringTok{'x64'}\NormalTok{,}
     \DataTypeTok{node_install_npm}\NormalTok{: }\StringTok{'true'}\NormalTok{,}
     \DataTypeTok{node_prefix}\NormalTok{: }\StringTok{''}\NormalTok{,}
     \DataTypeTok{node_shared_cares}\NormalTok{: }\StringTok{'false'}\NormalTok{,}
     \DataTypeTok{node_shared_http_parser}\NormalTok{: }\StringTok{'false'}\NormalTok{,}
     \DataTypeTok{node_shared_libuv}\NormalTok{: }\StringTok{'false'}\NormalTok{,}
     \DataTypeTok{node_shared_v8}\NormalTok{: }\StringTok{'false'}\NormalTok{,}
     \DataTypeTok{node_shared_zlib}\NormalTok{: }\StringTok{'false'}\NormalTok{,}
     \DataTypeTok{node_use_dtrace}\NormalTok{: }\StringTok{'false'}\NormalTok{,}
     \DataTypeTok{node_use_openssl}\NormalTok{: }\StringTok{'true'}\NormalTok{,}
     \DataTypeTok{node_shared_openssl}\NormalTok{: }\StringTok{'false'}\NormalTok{,}
     \DataTypeTok{strict_aliasing}\NormalTok{: }\StringTok{'true'}\NormalTok{,}
     \DataTypeTok{target_arch}\NormalTok{: }\StringTok{'x64'}\NormalTok{,}
     \DataTypeTok{v8_use_snapshot}\NormalTok{: }\StringTok{'true'} \NormalTok{\} \}}
\end{Highlighting}
\end{Shaded}

\subsection{process.kill(pid,
{[}signal{]})}\label{process.killpid-signal}

Send a signal to a process. \texttt{pid} is the process id and
\texttt{signal} is the string describing the signal to send. Signal
names are strings like `SIGINT' or `SIGHUP'. If omitted, the signal will
be `SIGTERM'. See \hyperref[processux5fsignalux5fevents]{Signal Events}
and kill(2) for more information.

Will throw an error if target does not exist, and as a special case, a
signal of \texttt{0} can be used to test for the existence of a process.

Note that just because the name of this function is
\texttt{process.kill}, it is really just a signal sender, like the
\texttt{kill} system call. The signal sent may do something other than
kill the target process.

Example of sending a signal to yourself:

\begin{Shaded}
\begin{Highlighting}[]
\OtherTok{process}\NormalTok{.}\FunctionTok{on}\NormalTok{(}\StringTok{'SIGHUP'}\NormalTok{, }\KeywordTok{function}\NormalTok{() \{}
  \OtherTok{console}\NormalTok{.}\FunctionTok{log}\NormalTok{(}\StringTok{'Got SIGHUP signal.'}\NormalTok{);}
\NormalTok{\});}

\FunctionTok{setTimeout}\NormalTok{(}\KeywordTok{function}\NormalTok{() \{}
  \OtherTok{console}\NormalTok{.}\FunctionTok{log}\NormalTok{(}\StringTok{'Exiting.'}\NormalTok{);}
  \OtherTok{process}\NormalTok{.}\FunctionTok{exit}\NormalTok{(}\DecValTok{0}\NormalTok{);}
\NormalTok{\}, }\DecValTok{100}\NormalTok{);}

\OtherTok{process}\NormalTok{.}\FunctionTok{kill}\NormalTok{(}\OtherTok{process}\NormalTok{.}\FunctionTok{pid}\NormalTok{, }\StringTok{'SIGHUP'}\NormalTok{);}
\end{Highlighting}
\end{Shaded}

Note: When SIGUSR1 is received by Node.js it starts the debugger, see
\hyperref[processux5fsignalux5fevents]{Signal Events}.

\subsection{process.pid}\label{process.pid}

The PID of the process.

\begin{Shaded}
\begin{Highlighting}[]
\OtherTok{console}\NormalTok{.}\FunctionTok{log}\NormalTok{(}\StringTok{'This process is pid '} \NormalTok{+ }\OtherTok{process}\NormalTok{.}\FunctionTok{pid}\NormalTok{);}
\end{Highlighting}
\end{Shaded}

\subsection{process.title}\label{process.title}

Getter/setter to set what is displayed in `ps'.

When used as a setter, the maximum length is platform-specific and
probably short.

On Linux and OS X, it's limited to the size of the binary name plus the
length of the command line arguments because it overwrites the argv
memory.

v0.8 allowed for longer process title strings by also overwriting the
environ memory but that was potentially insecure/confusing in some
(rather obscure) cases.

\subsection{process.arch}\label{process.arch}

What processor architecture you're running on: \texttt{'arm'},
\texttt{'ia32'}, or \texttt{'x64'}.

\begin{Shaded}
\begin{Highlighting}[]
\OtherTok{console}\NormalTok{.}\FunctionTok{log}\NormalTok{(}\StringTok{'This processor architecture is '} \NormalTok{+ }\OtherTok{process}\NormalTok{.}\FunctionTok{arch}\NormalTok{);}
\end{Highlighting}
\end{Shaded}

\subsection{process.platform}\label{process.platform}

What platform you're running on: \texttt{'darwin'}, \texttt{'freebsd'},
\texttt{'linux'}, \texttt{'sunos'} or \texttt{'win32'}

\begin{Shaded}
\begin{Highlighting}[]
\OtherTok{console}\NormalTok{.}\FunctionTok{log}\NormalTok{(}\StringTok{'This platform is '} \NormalTok{+ }\OtherTok{process}\NormalTok{.}\FunctionTok{platform}\NormalTok{);}
\end{Highlighting}
\end{Shaded}

\subsection{process.memoryUsage()}\label{process.memoryusage}

Returns an object describing the memory usage of the Node process
measured in bytes.

\begin{Shaded}
\begin{Highlighting}[]
\KeywordTok{var} \NormalTok{util = }\FunctionTok{require}\NormalTok{(}\StringTok{'util'}\NormalTok{);}

\OtherTok{console}\NormalTok{.}\FunctionTok{log}\NormalTok{(}\OtherTok{util}\NormalTok{.}\FunctionTok{inspect}\NormalTok{(}\OtherTok{process}\NormalTok{.}\FunctionTok{memoryUsage}\NormalTok{()));}
\end{Highlighting}
\end{Shaded}

This will generate:

\begin{Shaded}
\begin{Highlighting}[]
\NormalTok{\{ }\DataTypeTok{rss}\NormalTok{: }\DecValTok{4935680}\NormalTok{,}
  \DataTypeTok{heapTotal}\NormalTok{: }\DecValTok{1826816}\NormalTok{,}
  \DataTypeTok{heapUsed}\NormalTok{: }\DecValTok{650472} \NormalTok{\}}
\end{Highlighting}
\end{Shaded}

\texttt{heapTotal} and \texttt{heapUsed} refer to V8's memory usage.

\subsection{process.nextTick(callback)}\label{process.nexttickcallback}

\begin{itemize}
\itemsep1pt\parskip0pt\parsep0pt
\item
  \texttt{callback} \{Function\}
\end{itemize}

Once the current event loop turn runs to completion, call the callback
function.

This is \emph{not} a simple alias to \texttt{setTimeout(fn, 0)}, it's
much more efficient. It runs before any additional I/O events (including
timers) fire in subsequent ticks of the event loop.

\begin{Shaded}
\begin{Highlighting}[]
\OtherTok{console}\NormalTok{.}\FunctionTok{log}\NormalTok{(}\StringTok{'start'}\NormalTok{);}
\OtherTok{process}\NormalTok{.}\FunctionTok{nextTick}\NormalTok{(}\KeywordTok{function}\NormalTok{() \{}
  \OtherTok{console}\NormalTok{.}\FunctionTok{log}\NormalTok{(}\StringTok{'nextTick callback'}\NormalTok{);}
\NormalTok{\});}
\OtherTok{console}\NormalTok{.}\FunctionTok{log}\NormalTok{(}\StringTok{'scheduled'}\NormalTok{);}
\CommentTok{// Output:}
\CommentTok{// start}
\CommentTok{// scheduled}
\CommentTok{// nextTick callback}
\end{Highlighting}
\end{Shaded}

This is important in developing APIs where you want to give the user the
chance to assign event handlers after an object has been constructed,
but before any I/O has occurred.

\begin{Shaded}
\begin{Highlighting}[]
\KeywordTok{function} \FunctionTok{MyThing}\NormalTok{(options) \{}
  \KeywordTok{this}\NormalTok{.}\FunctionTok{setupOptions}\NormalTok{(options);}

  \OtherTok{process}\NormalTok{.}\FunctionTok{nextTick}\NormalTok{(}\KeywordTok{function}\NormalTok{() \{}
    \KeywordTok{this}\NormalTok{.}\FunctionTok{startDoingStuff}\NormalTok{();}
  \NormalTok{\}.}\FunctionTok{bind}\NormalTok{(}\KeywordTok{this}\NormalTok{));}
\NormalTok{\}}

\KeywordTok{var} \NormalTok{thing = }\KeywordTok{new} \FunctionTok{MyThing}\NormalTok{();}
\OtherTok{thing}\NormalTok{.}\FunctionTok{getReadyForStuff}\NormalTok{();}

\CommentTok{// thing.startDoingStuff() gets called now, not before.}
\end{Highlighting}
\end{Shaded}

It is very important for APIs to be either 100\% synchronous or 100\%
asynchronous. Consider this example:

\begin{Shaded}
\begin{Highlighting}[]
\CommentTok{// WARNING!  DO NOT USE!  BAD UNSAFE HAZARD!}
\KeywordTok{function} \FunctionTok{maybeSync}\NormalTok{(arg, cb) \{}
  \KeywordTok{if} \NormalTok{(arg) \{}
    \FunctionTok{cb}\NormalTok{();}
    \KeywordTok{return}\NormalTok{;}
  \NormalTok{\}}

  \OtherTok{fs}\NormalTok{.}\FunctionTok{stat}\NormalTok{(}\StringTok{'file'}\NormalTok{, cb);}
\NormalTok{\}}
\end{Highlighting}
\end{Shaded}

This API is hazardous. If you do this:

\begin{Shaded}
\begin{Highlighting}[]
\FunctionTok{maybeSync}\NormalTok{(}\KeywordTok{true}\NormalTok{, }\KeywordTok{function}\NormalTok{() \{}
  \FunctionTok{foo}\NormalTok{();}
\NormalTok{\});}
\FunctionTok{bar}\NormalTok{();}
\end{Highlighting}
\end{Shaded}

then it's not clear whether \texttt{foo()} or \texttt{bar()} will be
called first.

This approach is much better:

\begin{Shaded}
\begin{Highlighting}[]
\KeywordTok{function} \FunctionTok{definitelyAsync}\NormalTok{(arg, cb) \{}
  \KeywordTok{if} \NormalTok{(arg) \{}
    \OtherTok{process}\NormalTok{.}\FunctionTok{nextTick}\NormalTok{(cb);}
    \KeywordTok{return}\NormalTok{;}
  \NormalTok{\}}

  \OtherTok{fs}\NormalTok{.}\FunctionTok{stat}\NormalTok{(}\StringTok{'file'}\NormalTok{, cb);}
\NormalTok{\}}
\end{Highlighting}
\end{Shaded}

Note: the nextTick queue is completely drained on each pass of the event
loop \textbf{before} additional I/O is processed. As a result,
recursively setting nextTick callbacks will block any I/O from
happening, just like a \texttt{while(true);} loop.

\subsection{process.umask({[}mask{]})}\label{process.umaskmask}

Sets or reads the process's file mode creation mask. Child processes
inherit the mask from the parent process. Returns the old mask if
\texttt{mask} argument is given, otherwise returns the current mask.

\begin{Shaded}
\begin{Highlighting}[]
\KeywordTok{var} \NormalTok{oldmask, newmask = }\DecValTok{0644}\NormalTok{;}

\NormalTok{oldmask = }\OtherTok{process}\NormalTok{.}\FunctionTok{umask}\NormalTok{(newmask);}
\OtherTok{console}\NormalTok{.}\FunctionTok{log}\NormalTok{(}\StringTok{'Changed umask from: '} \NormalTok{+ }\OtherTok{oldmask}\NormalTok{.}\FunctionTok{toString}\NormalTok{(}\DecValTok{8}\NormalTok{) +}
            \StringTok{' to '} \NormalTok{+ }\OtherTok{newmask}\NormalTok{.}\FunctionTok{toString}\NormalTok{(}\DecValTok{8}\NormalTok{));}
\end{Highlighting}
\end{Shaded}

\subsection{process.uptime()}\label{process.uptime}

Number of seconds Node has been running.

\subsection{process.hrtime()}\label{process.hrtime}

Returns the current high-resolution real time in a
\texttt{{[}seconds, nanoseconds{]}} tuple Array. It is relative to an
arbitrary time in the past. It is not related to the time of day and
therefore not subject to clock drift. The primary use is for measuring
performance between intervals.

You may pass in the result of a previous call to
\texttt{process.hrtime()} to get a diff reading, useful for benchmarks
and measuring intervals:

\begin{Shaded}
\begin{Highlighting}[]
\KeywordTok{var} \NormalTok{time = }\OtherTok{process}\NormalTok{.}\FunctionTok{hrtime}\NormalTok{();}
\CommentTok{// [ 1800216, 25 ]}

\FunctionTok{setTimeout}\NormalTok{(}\KeywordTok{function}\NormalTok{() \{}
  \KeywordTok{var} \NormalTok{diff = }\OtherTok{process}\NormalTok{.}\FunctionTok{hrtime}\NormalTok{(time);}
  \CommentTok{// [ 1, 552 ]}

  \OtherTok{console}\NormalTok{.}\FunctionTok{log}\NormalTok{(}\StringTok{'benchmark took %d nanoseconds'}\NormalTok{, diff[}\DecValTok{0}\NormalTok{] * }\FloatTok{1e9} \NormalTok{+ diff[}\DecValTok{1}\NormalTok{]);}
  \CommentTok{// benchmark took 1000000527 nanoseconds}
\NormalTok{\}, }\DecValTok{1000}\NormalTok{);}
\end{Highlighting}
\end{Shaded}

