\section{Synopsis}

An example of a \href{http.html}{web server} written with Node which
responds with `Hello World':

\begin{Shaded}
\begin{Highlighting}[]
\KeywordTok{var} \NormalTok{http = require(}\CharTok{'http'}\NormalTok{);}

\KeywordTok{http}\NormalTok{.}\FunctionTok{createServer}\NormalTok{(}\KeywordTok{function} \NormalTok{(request, response) \{}
  \KeywordTok{response}\NormalTok{.}\FunctionTok{writeHead}\NormalTok{(}\DecValTok{200}\NormalTok{, \{}\CharTok{'Content-Type'}\NormalTok{: }\CharTok{'text/plain'}\NormalTok{\});}
  \KeywordTok{response}\NormalTok{.}\FunctionTok{end}\NormalTok{(}\CharTok{'Hello World\textbackslash{}n'}\NormalTok{);}
\NormalTok{\}).}\FunctionTok{listen}\NormalTok{(}\DecValTok{8124}\NormalTok{);}

\KeywordTok{console}\NormalTok{.}\FunctionTok{log}\NormalTok{(}\CharTok{'Server running at http://127.0.0.1:8124/'}\NormalTok{);}
\end{Highlighting}
\end{Shaded}

To run the server, put the code into a file called \texttt{example.js}
and execute it with the node program

\begin{Shaded}
\begin{Highlighting}[]
\NormalTok{> node }\KeywordTok{example}\NormalTok{.}\FunctionTok{js}
\FunctionTok{Server} \NormalTok{running at }\DataTypeTok{http}\NormalTok{:}\CommentTok{//127.0.0.1:8124/}
\end{Highlighting}
\end{Shaded}

All of the examples in the documentation can be run similarly.
