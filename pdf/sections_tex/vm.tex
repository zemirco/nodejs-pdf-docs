\section{Executing JavaScript}

\begin{Shaded}
\begin{Highlighting}[]
\DataTypeTok{Stability}\NormalTok{: }\DecValTok{2} \NormalTok{- }\KeywordTok{Unstable}\NormalTok{. See Caveats, }\KeywordTok{below}\NormalTok{.}
\end{Highlighting}
\end{Shaded}

You can access this module with:

\begin{Shaded}
\begin{Highlighting}[]
\KeywordTok{var} \NormalTok{vm = require(}\CharTok{'vm'}\NormalTok{);}
\end{Highlighting}
\end{Shaded}

JavaScript code can be compiled and run immediately or compiled, saved,
and run later.

\subsection{Caveats}

The \texttt{vm} module has many known issues and edge cases. If you run
into issues or unexpected behavior, please consult
\href{https://github.com/joyent/node/search?q=vm+state\%3Aopen\&type=Issues}{the
open issues on GitHub}. Some of the biggest problems are described
below.

\subsubsection{Sandboxes}

The \texttt{sandbox} argument to \texttt{vm.runInNewContext} and
\texttt{vm.createContext}, along with the \texttt{initSandbox} argument
to \texttt{vm.createContext}, do not behave as one might normally expect
and their behavior varies between different versions of Node.

The key issue to be aware of is that V8 provides no way to directly
control the global object used within a context. As a result, while
properties of your \texttt{sandbox} object will be available in the
context, any properties from the \texttt{prototype}s of the
\texttt{sandbox} may not be available. Furthermore, the \texttt{this}
expression within the global scope of the context evaluates to the empty
object (\texttt{\{\}}) instead of to your sandbox.

Your sandbox's properties are also not shared directly with the script.
Instead, the properties of the sandbox are copied into the context at
the beginning of execution, and then after execution, the properties are
copied back out in an attempt to propagate any changes.

\subsubsection{Globals}

Properties of the global object, like \texttt{Array} and
\texttt{String}, have different values inside of a context. This means
that common expressions like \texttt{{[}{]} instanceof Array} or
\texttt{Object.getPrototypeOf({[}{]}) === Array.prototype} may not
produce expected results when used inside of scripts evaluated via the
\texttt{vm} module.

Some of these problems have known workarounds listed in the issues for
\texttt{vm} on GitHub. for example, \texttt{Array.isArray} works around
the example problem with \texttt{Array}.

\subsection{vm.runInThisContext(code, {[}filename{]})}

\texttt{vm.runInThisContext()} compiles \texttt{code}, runs it and
returns the result. Running code does not have access to local scope.
\texttt{filename} is optional, it's used only in stack traces.

Example of using \texttt{vm.runInThisContext} and \texttt{eval} to run
the same code:

\begin{Shaded}
\begin{Highlighting}[]
\KeywordTok{var} \NormalTok{localVar = }\DecValTok{123}\NormalTok{,}
    \NormalTok{usingscript, evaled,}
    \NormalTok{vm = require(}\CharTok{'vm'}\NormalTok{);}

\NormalTok{usingscript = }\KeywordTok{vm}\NormalTok{.}\FunctionTok{runInThisContext}\NormalTok{(}\CharTok{'localVar = 1;'}\NormalTok{,}
  \CharTok{'myfile.vm'}\NormalTok{);}
\KeywordTok{console}\NormalTok{.}\FunctionTok{log}\NormalTok{(}\CharTok{'localVar: '} \NormalTok{+ localVar + }\CharTok{', usingscript: '} \NormalTok{+}
  \NormalTok{usingscript);}
\NormalTok{evaled = eval(}\CharTok{'localVar = 1;'}\NormalTok{);}
\KeywordTok{console}\NormalTok{.}\FunctionTok{log}\NormalTok{(}\CharTok{'localVar: '} \NormalTok{+ localVar + }\CharTok{', evaled: '} \NormalTok{+}
  \NormalTok{evaled);}

\CommentTok{// localVar: 123, usingscript: 1}
\CommentTok{// localVar: 1, evaled: 1}
\end{Highlighting}
\end{Shaded}

\texttt{vm.runInThisContext} does not have access to the local scope, so
\texttt{localVar} is unchanged. \texttt{eval} does have access to the
local scope, so \texttt{localVar} is changed.

In case of syntax error in \texttt{code}, \texttt{vm.runInThisContext}
emits the syntax error to stderr and throws an exception.

\subsection{vm.runInNewContext(code, {[}sandbox{]}, {[}filename{]},
{[}timeout{]})}

\texttt{vm.runInNewContext} compiles \texttt{code}, then runs it in
\texttt{sandbox} and returns the result. Running code does not have
access to local scope. The object \texttt{sandbox} will be used as the
global object for \texttt{code}. \texttt{sandbox} and \texttt{filename}
are optional, \texttt{filename} is only used in stack traces.
\texttt{timeout} specifies an optional number of milliseconds to execute
\texttt{code} before terminating execution. If execution is terminated,
\texttt{null} will be thrown.

Example: compile and execute code that increments a global variable and
sets a new one. These globals are contained in the sandbox.

\begin{Shaded}
\begin{Highlighting}[]
\KeywordTok{var} \NormalTok{util = require(}\CharTok{'util'}\NormalTok{),}
    \NormalTok{vm = require(}\CharTok{'vm'}\NormalTok{),}
    \NormalTok{sandbox = \{}
      \DataTypeTok{animal}\NormalTok{: }\CharTok{'cat'}\NormalTok{,}
      \DataTypeTok{count}\NormalTok{: }\DecValTok{2}
    \NormalTok{\};}

\KeywordTok{vm}\NormalTok{.}\FunctionTok{runInNewContext}\NormalTok{(}\CharTok{'count += 1; name = "kitty"'}\NormalTok{, sandbox, }\CharTok{'myfile.vm'}\NormalTok{);}
\KeywordTok{console}\NormalTok{.}\FunctionTok{log}\NormalTok{(}\KeywordTok{util}\NormalTok{.}\FunctionTok{inspect}\NormalTok{(sandbox));}

\CommentTok{// \{ animal: 'cat', count: 3, name: 'kitty' \}}
\end{Highlighting}
\end{Shaded}

Note that running untrusted code is a tricky business requiring great
care. To prevent accidental global variable leakage,
\texttt{vm.runInNewContext} is quite useful, but safely running
untrusted code requires a separate process.

In case of syntax error in \texttt{code}, \texttt{vm.runInNewContext}
emits the syntax error to stderr and throws an exception.

\subsection{vm.runInContext(code, context, {[}filename{]},
{[}timeout{]})}

\texttt{vm.runInContext} compiles \texttt{code}, then runs it in
\texttt{context} and returns the result. A (V8) context comprises a
global object, together with a set of built-in objects and functions.
Running code does not have access to local scope and the global object
held within \texttt{context} will be used as the global object for
\texttt{code}. \texttt{filename} is optional, it's used only in stack
traces. \texttt{timeout} specifies an optional number of milliseconds to
execute \texttt{code} before terminating execution. If execution is
terminated, \texttt{null} will be thrown.

Example: compile and execute code in a existing context.

\begin{Shaded}
\begin{Highlighting}[]
\KeywordTok{var} \NormalTok{util = require(}\CharTok{'util'}\NormalTok{),}
    \NormalTok{vm = require(}\CharTok{'vm'}\NormalTok{),}
    \NormalTok{initSandbox = \{}
      \DataTypeTok{animal}\NormalTok{: }\CharTok{'cat'}\NormalTok{,}
      \DataTypeTok{count}\NormalTok{: }\DecValTok{2}
    \NormalTok{\},}
    \NormalTok{context = }\KeywordTok{vm}\NormalTok{.}\FunctionTok{createContext}\NormalTok{(initSandbox);}

\KeywordTok{vm}\NormalTok{.}\FunctionTok{runInContext}\NormalTok{(}\CharTok{'count += 1; name = "CATT"'}\NormalTok{, context, }\CharTok{'myfile.vm'}\NormalTok{);}
\KeywordTok{console}\NormalTok{.}\FunctionTok{log}\NormalTok{(}\KeywordTok{util}\NormalTok{.}\FunctionTok{inspect}\NormalTok{(context));}

\CommentTok{// \{ animal: 'cat', count: 3, name: 'CATT' \}}
\end{Highlighting}
\end{Shaded}

Note that \texttt{createContext} will perform a shallow clone of the
supplied sandbox object in order to initialize the global object of the
freshly constructed context.

Note that running untrusted code is a tricky business requiring great
care. To prevent accidental global variable leakage,
\texttt{vm.runInContext} is quite useful, but safely running untrusted
code requires a separate process.

In case of syntax error in \texttt{code}, \texttt{vm.runInContext} emits
the syntax error to stderr and throws an exception.

\subsection{vm.createContext({[}initSandbox{]})}

\texttt{vm.createContext} creates a new context which is suitable for
use as the 2nd argument of a subsequent call to
\texttt{vm.runInContext}. A (V8) context comprises a global object
together with a set of build-in objects and functions. The optional
argument \texttt{initSandbox} will be shallow-copied to seed the initial
contents of the global object used by the context.

\subsection{vm.createScript(code, {[}filename{]})}

\texttt{createScript} compiles \texttt{code} but does not run it.
Instead, it returns a \texttt{vm.Script} object representing this
compiled code. This script can be run later many times using methods
below. The returned script is not bound to any global object. It is
bound before each run, just for that run. \texttt{filename} is optional,
it's only used in stack traces.

In case of syntax error in \texttt{code}, \texttt{createScript} prints
the syntax error to stderr and throws an exception.

\subsection{Class: Script}

A class for running scripts. Returned by vm.createScript.

\subsubsection{script.runInThisContext({[}timeout{]})}

Similar to \texttt{vm.runInThisContext} but a method of a precompiled
\texttt{Script} object. \texttt{script.runInThisContext} runs the code
of \texttt{script} and returns the result. Running code does not have
access to local scope, but does have access to the \texttt{global}
object (v8: in actual context). \texttt{timeout} specifies an optional
number of milliseconds to execute \texttt{code} before terminating
execution. If execution is terminated, \texttt{null} will be thrown.

Example of using \texttt{script.runInThisContext} to compile code once
and run it multiple times:

\begin{Shaded}
\begin{Highlighting}[]
\KeywordTok{var} \NormalTok{vm = require(}\CharTok{'vm'}\NormalTok{);}

\NormalTok{globalVar = }\DecValTok{0}\NormalTok{;}

\KeywordTok{var} \NormalTok{script = }\KeywordTok{vm}\NormalTok{.}\FunctionTok{createScript}\NormalTok{(}\CharTok{'globalVar += 1'}\NormalTok{, }\CharTok{'myfile.vm'}\NormalTok{);}

\KeywordTok{for} \NormalTok{(}\KeywordTok{var} \NormalTok{i = }\DecValTok{0}\NormalTok{; i < }\DecValTok{1000} \NormalTok{; i += }\DecValTok{1}\NormalTok{) \{}
  \KeywordTok{script}\NormalTok{.}\FunctionTok{runInThisContext}\NormalTok{();}
\NormalTok{\}}

\KeywordTok{console}\NormalTok{.}\FunctionTok{log}\NormalTok{(globalVar);}

\CommentTok{// 1000}
\end{Highlighting}
\end{Shaded}

\subsubsection{script.runInNewContext({[}sandbox{]}, {[}timeout{]})}

Similar to \texttt{vm.runInNewContext} a method of a precompiled
\texttt{Script} object. \texttt{script.runInNewContext} runs the code of
\texttt{script} with \texttt{sandbox} as the global object and returns
the result. Running code does not have access to local scope.
\texttt{sandbox} is optional. \texttt{timeout} specifies an optional
number of milliseconds to execute \texttt{code} before terminating
execution. If execution is terminated, \texttt{null} will be thrown.

Example: compile code that increments a global variable and sets one,
then execute this code multiple times. These globals are contained in
the sandbox.

\begin{Shaded}
\begin{Highlighting}[]
\KeywordTok{var} \NormalTok{util = require(}\CharTok{'util'}\NormalTok{),}
    \NormalTok{vm = require(}\CharTok{'vm'}\NormalTok{),}
    \NormalTok{sandbox = \{}
      \DataTypeTok{animal}\NormalTok{: }\CharTok{'cat'}\NormalTok{,}
      \DataTypeTok{count}\NormalTok{: }\DecValTok{2}
    \NormalTok{\};}

\KeywordTok{var} \NormalTok{script = }\KeywordTok{vm}\NormalTok{.}\FunctionTok{createScript}\NormalTok{(}\CharTok{'count += 1; name = "kitty"'}\NormalTok{, }\CharTok{'myfile.vm'}\NormalTok{);}

\KeywordTok{for} \NormalTok{(}\KeywordTok{var} \NormalTok{i = }\DecValTok{0}\NormalTok{; i < }\DecValTok{10} \NormalTok{; i += }\DecValTok{1}\NormalTok{) \{}
  \KeywordTok{script}\NormalTok{.}\FunctionTok{runInNewContext}\NormalTok{(sandbox);}
\NormalTok{\}}

\KeywordTok{console}\NormalTok{.}\FunctionTok{log}\NormalTok{(}\KeywordTok{util}\NormalTok{.}\FunctionTok{inspect}\NormalTok{(sandbox));}

\CommentTok{// \{ animal: 'cat', count: 12, name: 'kitty' \}}
\end{Highlighting}
\end{Shaded}

Note that running untrusted code is a tricky business requiring great
care. To prevent accidental global variable leakage,
\texttt{script.runInNewContext} is quite useful, but safely running
untrusted code requires a separate process.
