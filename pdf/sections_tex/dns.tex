\section{DNS}\label{dns}

\begin{Shaded}
\begin{Highlighting}[]
\NormalTok{Stability: }\DecValTok{3} \NormalTok{- Stable}
\end{Highlighting}
\end{Shaded}

Use \texttt{require('dns')} to access this module. All methods in the
dns module use C-Ares except for \texttt{dns.lookup} which uses
\texttt{getaddrinfo(3)} in a thread pool. C-Ares is much faster than
\texttt{getaddrinfo} but the system resolver is more consistent with how
other programs operate. When a user does
\texttt{net.connect(80, 'google.com')} or
\texttt{http.get(\{ host: 'google.com' \})} the \texttt{dns.lookup}
method is used. Users who need to do a large number of lookups quickly
should use the methods that go through C-Ares.

Here is an example which resolves \texttt{'www.google.com'} then reverse
resolves the IP addresses which are returned.

\begin{Shaded}
\begin{Highlighting}[]
\KeywordTok{var} \NormalTok{dns = }\FunctionTok{require}\NormalTok{(}\StringTok{'dns'}\NormalTok{);}

\OtherTok{dns}\NormalTok{.}\FunctionTok{resolve4}\NormalTok{(}\StringTok{'www.google.com'}\NormalTok{, }\KeywordTok{function} \NormalTok{(err, addresses) \{}
  \KeywordTok{if} \NormalTok{(err) }\KeywordTok{throw} \NormalTok{err;}

  \OtherTok{console}\NormalTok{.}\FunctionTok{log}\NormalTok{(}\StringTok{'addresses: '} \NormalTok{+ }\OtherTok{JSON}\NormalTok{.}\FunctionTok{stringify}\NormalTok{(addresses));}

  \OtherTok{addresses}\NormalTok{.}\FunctionTok{forEach}\NormalTok{(}\KeywordTok{function} \NormalTok{(a) \{}
    \OtherTok{dns}\NormalTok{.}\FunctionTok{reverse}\NormalTok{(a, }\KeywordTok{function} \NormalTok{(err, hostnames) \{}
      \KeywordTok{if} \NormalTok{(err) \{}
        \KeywordTok{throw} \NormalTok{err;}
      \NormalTok{\}}

      \OtherTok{console}\NormalTok{.}\FunctionTok{log}\NormalTok{(}\StringTok{'reverse for '} \NormalTok{+ a + }\StringTok{': '} \NormalTok{+ }\OtherTok{JSON}\NormalTok{.}\FunctionTok{stringify}\NormalTok{(hostnames));}
    \NormalTok{\});}
  \NormalTok{\});}
\NormalTok{\});}
\end{Highlighting}
\end{Shaded}

\subsection{dns.lookup(hostname, {[}options{]},
callback)}\label{dns.lookuphostname-options-callback}

Resolves a hostname (e.g. \texttt{'google.com'}) into the first found A
(IPv4) or AAAA (IPv6) record. \texttt{options} can be an object or
integer. If \texttt{options} is not provided, then IP v4 and v6
addresses are both valid. If \texttt{options} is an integer, then it
must be \texttt{4} or \texttt{6}.

Alternatively, \texttt{options} can be an object containing two
properties, \texttt{family} and \texttt{hints}. Both properties are
optional. If \texttt{family} is provided, it must be the integer
\texttt{4} or \texttt{6}. If \texttt{family} is not provided then IP v4
and v6 addresses are accepted. The \texttt{hints} field, if present,
should be one or more of the supported \texttt{getaddrinfo} flags. If
\texttt{hints} is not provided, then no flags are passed to
\texttt{getaddrinfo}. Multiple flags can be passed through
\texttt{hints} by logically \texttt{OR}ing their values. An example
usage of \texttt{options} is shown below.

\begin{verbatim}
{
  family: 4,
  hints: dns.ADDRCONFIG | dns.V4MAPPED
}
\end{verbatim}

See \hyperref[dnsux5fsupportedux5fgetaddrinfoux5fflags]{supported
\texttt{getaddrinfo} flags} below for more information on supported
flags.

The callback has arguments \texttt{(err, address, family)}. The
\texttt{address} argument is a string representation of a IP v4 or v6
address. The \texttt{family} argument is either the integer 4 or 6 and
denotes the family of \texttt{address} (not necessarily the value
initially passed to \texttt{lookup}).

On error, \texttt{err} is an \texttt{Error} object, where
\texttt{err.code} is the error code. Keep in mind that \texttt{err.code}
will be set to \texttt{'ENOENT'} not only when the hostname does not
exist but also when the lookup fails in other ways such as no available
file descriptors.

\section{dns.lookupService(address, port,
callback)}\label{dns.lookupserviceaddress-port-callback}

Resolves the given address and port into a hostname and service using
\texttt{getnameinfo}.

The callback has arguments \texttt{(err, hostname, service)}. The
\texttt{hostname} and \texttt{service} arguments are strings (e.g.
\texttt{'localhost'} and \texttt{'http'} respectively).

On error, \texttt{err} is an \texttt{Error} object, where
\texttt{err.code} is the error code.

\subsection{dns.resolve(hostname, {[}rrtype{]},
callback)}\label{dns.resolvehostname-rrtype-callback}

Resolves a hostname (e.g. \texttt{'google.com'}) into an array of the
record types specified by rrtype.

Valid rrtypes are:

\begin{itemize}
\itemsep1pt\parskip0pt\parsep0pt
\item
  \texttt{'A'} (IPV4 addresses, default)
\item
  \texttt{'AAAA'} (IPV6 addresses)
\item
  \texttt{'MX'} (mail exchange records)
\item
  \texttt{'TXT'} (text records)
\item
  \texttt{'SRV'} (SRV records)
\item
  \texttt{'PTR'} (used for reverse IP lookups)
\item
  \texttt{'NS'} (name server records)
\item
  \texttt{'CNAME'} (canonical name records)
\item
  \texttt{'SOA'} (start of authority record)
\end{itemize}

The callback has arguments \texttt{(err, addresses)}. The type of each
item in \texttt{addresses} is determined by the record type, and
described in the documentation for the corresponding lookup methods
below.

On error, \texttt{err} is an \texttt{Error} object, where
\texttt{err.code} is one of the error codes listed below.

\subsection{dns.resolve4(hostname,
callback)}\label{dns.resolve4hostname-callback}

The same as \texttt{dns.resolve()}, but only for IPv4 queries
(\texttt{A} records). \texttt{addresses} is an array of IPv4 addresses
(e.g. \texttt{{[}'74.125.79.104', '74.125.79.105', '74.125.79.106'{]}}).

\subsection{dns.resolve6(hostname,
callback)}\label{dns.resolve6hostname-callback}

The same as \texttt{dns.resolve4()} except for IPv6 queries (an
\texttt{AAAA} query).

\subsection{dns.resolveMx(hostname,
callback)}\label{dns.resolvemxhostname-callback}

The same as \texttt{dns.resolve()}, but only for mail exchange queries
(\texttt{MX} records).

\texttt{addresses} is an array of MX records, each with a priority and
an exchange attribute (e.g.
\texttt{{[}\{'priority': 10, 'exchange': 'mx.example.com'\},...{]}}).

\subsection{dns.resolveTxt(hostname,
callback)}\label{dns.resolvetxthostname-callback}

The same as \texttt{dns.resolve()}, but only for text queries
(\texttt{TXT} records). \texttt{addresses} is an 2-d array of the text
records available for \texttt{hostname} (e.g.,
\texttt{{[} {[}'v=spf1 ip4:0.0.0.0 ', '\textasciitilde{}all' {]} {]}}).
Each sub-array contains TXT chunks of one record. Depending on the use
case, the could be either joined together or treated separately.

\subsection{dns.resolveSrv(hostname,
callback)}\label{dns.resolvesrvhostname-callback}

The same as \texttt{dns.resolve()}, but only for service records
(\texttt{SRV} records). \texttt{addresses} is an array of the SRV
records available for \texttt{hostname}. Properties of SRV records are
priority, weight, port, and name (e.g.,
\texttt{{[}\{'priority': 10, 'weight': 5, 'port': 21223, 'name': 'service.example.com'\}, ...{]}}).

\subsection{dns.resolveSoa(hostname,
callback)}\label{dns.resolvesoahostname-callback}

The same as \texttt{dns.resolve()}, but only for start of authority
record queries (\texttt{SOA} record).

\texttt{addresses} is an object with the following structure:

\begin{verbatim}
{
  nsname: 'ns.example.com',
  hostmaster: 'root.example.com',
  serial: 2013101809,
  refresh: 10000,
  retry: 2400,
  expire: 604800,
  minttl: 3600
}
\end{verbatim}

\subsection{dns.resolveNs(hostname,
callback)}\label{dns.resolvenshostname-callback}

The same as \texttt{dns.resolve()}, but only for name server records
(\texttt{NS} records). \texttt{addresses} is an array of the name server
records available for \texttt{hostname} (e.g.,
\texttt{{[}'ns1.example.com', 'ns2.example.com'{]}}).

\subsection{dns.resolveCname(hostname,
callback)}\label{dns.resolvecnamehostname-callback}

The same as \texttt{dns.resolve()}, but only for canonical name records
(\texttt{CNAME} records). \texttt{addresses} is an array of the
canonical name records available for \texttt{hostname} (e.g.,
\texttt{{[}'bar.example.com'{]}}).

\subsection{dns.reverse(ip, callback)}\label{dns.reverseip-callback}

Reverse resolves an ip address to an array of hostnames.

The callback has arguments \texttt{(err, hostnames)}.

On error, \texttt{err} is an \texttt{Error} object, where
\texttt{err.code} is one of the error codes listed below.

\subsection{dns.getServers()}\label{dns.getservers}

Returns an array of IP addresses as strings that are currently being
used for resolution

\subsection{dns.setServers(servers)}\label{dns.setserversservers}

Given an array of IP addresses as strings, set them as the servers to
use for resolving

If you specify a port with the address it will be stripped, as the
underlying library doesn't support that.

This will throw if you pass invalid input.

\subsection{Error codes}\label{error-codes}

Each DNS query can return one of the following error codes:

\begin{itemize}
\itemsep1pt\parskip0pt\parsep0pt
\item
  \texttt{dns.NODATA}: DNS server returned answer with no data.
\item
  \texttt{dns.FORMERR}: DNS server claims query was misformatted.
\item
  \texttt{dns.SERVFAIL}: DNS server returned general failure.
\item
  \texttt{dns.NOTFOUND}: Domain name not found.
\item
  \texttt{dns.NOTIMP}: DNS server does not implement requested
  operation.
\item
  \texttt{dns.REFUSED}: DNS server refused query.
\item
  \texttt{dns.BADQUERY}: Misformatted DNS query.
\item
  \texttt{dns.BADNAME}: Misformatted hostname.
\item
  \texttt{dns.BADFAMILY}: Unsupported address family.
\item
  \texttt{dns.BADRESP}: Misformatted DNS reply.
\item
  \texttt{dns.CONNREFUSED}: Could not contact DNS servers.
\item
  \texttt{dns.TIMEOUT}: Timeout while contacting DNS servers.
\item
  \texttt{dns.EOF}: End of file.
\item
  \texttt{dns.FILE}: Error reading file.
\item
  \texttt{dns.NOMEM}: Out of memory.
\item
  \texttt{dns.DESTRUCTION}: Channel is being destroyed.
\item
  \texttt{dns.BADSTR}: Misformatted string.
\item
  \texttt{dns.BADFLAGS}: Illegal flags specified.
\item
  \texttt{dns.NONAME}: Given hostname is not numeric.
\item
  \texttt{dns.BADHINTS}: Illegal hints flags specified.
\item
  \texttt{dns.NOTINITIALIZED}: c-ares library initialization not yet
  performed.
\item
  \texttt{dns.LOADIPHLPAPI}: Error loading iphlpapi.dll.
\item
  \texttt{dns.ADDRGETNETWORKPARAMS}: Could not find GetNetworkParams
  function.
\item
  \texttt{dns.CANCELLED}: DNS query cancelled.
\end{itemize}

\subsection{Supported getaddrinfo
flags}\label{supported-getaddrinfo-flags}

The following flags can be passed as hints to \texttt{dns.lookup}.

\begin{itemize}
\itemsep1pt\parskip0pt\parsep0pt
\item
  \texttt{dns.ADDRCONFIG}: Returned address types are determined by the
  types of addresses supported by the current system. For example, IPv4
  addresses are only returned if the current system has at least one
  IPv4 address configured. Loopback addresses are not considered.
\item
  \texttt{dns.V4MAPPED}: If the IPv6 family was specified, but no IPv6
  addresses were found, then return IPv4 mapped IPv6 addresses.
\end{itemize}
