\section{Smalloc}\label{smalloc}

\begin{Shaded}
\begin{Highlighting}[]
\NormalTok{Stability: }\DecValTok{1} \NormalTok{- Experimental}
\end{Highlighting}
\end{Shaded}

\subsection{Class: smalloc}\label{class-smalloc}

Buffers are backed by a simple allocator that only handles the
assignation of external raw memory. Smalloc exposes that functionality.

\subsubsection{smalloc.alloc(length{[}, receiver{]}{[},
type{]})}\label{smalloc.alloclength-receiver-type}

\begin{itemize}
\itemsep1pt\parskip0pt\parsep0pt
\item
  \texttt{length} \{Number\} \texttt{\textless{}=\ smalloc.kMaxLength}
\item
  \texttt{receiver} \{Object\} Default: \texttt{new\ Object}
\item
  \texttt{type} \{Enum\} Default: \texttt{Uint8}
\end{itemize}

Returns \texttt{receiver} with allocated external array data. If no
\texttt{receiver} is passed then a new Object will be created and
returned.

This can be used to create your own Buffer-like classes. No other
properties are set, so the user will need to keep track of other
necessary information (e.g. \texttt{length} of the allocation).

\begin{Shaded}
\begin{Highlighting}[]
\KeywordTok{function} \FunctionTok{SimpleData}\NormalTok{(n) \{}
  \KeywordTok{this}\NormalTok{.}\FunctionTok{length} \NormalTok{= n;}
  \OtherTok{smalloc}\NormalTok{.}\FunctionTok{alloc}\NormalTok{(}\KeywordTok{this}\NormalTok{.}\FunctionTok{length}\NormalTok{, }\KeywordTok{this}\NormalTok{);}
\NormalTok{\}}

\OtherTok{SimpleData}\NormalTok{.}\FunctionTok{prototype} \NormalTok{= \{ }\CommentTok{/* ... */} \NormalTok{\};}
\end{Highlighting}
\end{Shaded}

It only checks if the \texttt{receiver} is an Object, and also not an
Array. Because of this it is possible to allocate external array data to
more than a plain Object.

\begin{Shaded}
\begin{Highlighting}[]
\KeywordTok{function} \FunctionTok{allocMe}\NormalTok{() \{ \}}
\OtherTok{smalloc}\NormalTok{.}\FunctionTok{alloc}\NormalTok{(}\DecValTok{3}\NormalTok{, allocMe);}

\CommentTok{// \{ [Function allocMe] '0': 0, '1': 0, '2': 0 \}}
\end{Highlighting}
\end{Shaded}

v8 does not support allocating external array data to an Array, and if
passed will throw.

It's possible to specify the type of external array data you would like.
All possible options are listed in \texttt{smalloc.Types}. Example
usage:

\begin{Shaded}
\begin{Highlighting}[]
\KeywordTok{var} \NormalTok{doubleArr = }\OtherTok{smalloc}\NormalTok{.}\FunctionTok{alloc}\NormalTok{(}\DecValTok{3}\NormalTok{, }\OtherTok{smalloc}\NormalTok{.}\OtherTok{Types}\NormalTok{.}\FunctionTok{Double}\NormalTok{);}

\KeywordTok{for} \NormalTok{(}\KeywordTok{var} \NormalTok{i = }\DecValTok{0}\NormalTok{; i < }\DecValTok{3}\NormalTok{; i++)}
  \NormalTok{doubleArr = i / }\DecValTok{10}\NormalTok{;}

\CommentTok{// \{ '0': 0, '1': 0.1, '2': 0.2 \}}
\end{Highlighting}
\end{Shaded}

It is not possible to freeze, seal and prevent extensions of objects
with external data using \texttt{Object.freeze}, \texttt{Object.seal}
and \texttt{Object.preventExtensions} respectively.

\subsubsection{smalloc.copyOnto(source, sourceStart, dest, destStart,
copyLength);}\label{smalloc.copyontosource-sourcestart-dest-deststart-copylength}

\begin{itemize}
\itemsep1pt\parskip0pt\parsep0pt
\item
  \texttt{source} \{Object\} with external array allocation
\item
  \texttt{sourceStart} \{Number\} Position to begin copying from
\item
  \texttt{dest} \{Object\} with external array allocation
\item
  \texttt{destStart} \{Number\} Position to begin copying onto
\item
  \texttt{copyLength} \{Number\} Length of copy
\end{itemize}

Copy memory from one external array allocation to another. No arguments
are optional, and any violation will throw.

\begin{Shaded}
\begin{Highlighting}[]
\KeywordTok{var} \NormalTok{a = }\OtherTok{smalloc}\NormalTok{.}\FunctionTok{alloc}\NormalTok{(}\DecValTok{4}\NormalTok{);}
\KeywordTok{var} \NormalTok{b = }\OtherTok{smalloc}\NormalTok{.}\FunctionTok{alloc}\NormalTok{(}\DecValTok{4}\NormalTok{);}

\KeywordTok{for} \NormalTok{(}\KeywordTok{var} \NormalTok{i = }\DecValTok{0}\NormalTok{; i < }\DecValTok{4}\NormalTok{; i++) \{}
  \NormalTok{a[i] = i;}
  \NormalTok{b[i] = i * }\DecValTok{2}\NormalTok{;}
\NormalTok{\}}

\CommentTok{// \{ '0': 0, '1': 1, '2': 2, '3': 3 \}}
\CommentTok{// \{ '0': 0, '1': 2, '2': 4, '3': 6 \}}

\OtherTok{smalloc}\NormalTok{.}\FunctionTok{copyOnto}\NormalTok{(b, }\DecValTok{2}\NormalTok{, a, }\DecValTok{0}\NormalTok{, }\DecValTok{2}\NormalTok{);}

\CommentTok{// \{ '0': 4, '1': 6, '2': 2, '3': 3 \}}
\end{Highlighting}
\end{Shaded}

\texttt{copyOnto} automatically detects the length of the allocation
internally, so no need to set any additional properties for this to
work.

\subsubsection{smalloc.dispose(obj)}\label{smalloc.disposeobj}

\begin{itemize}
\itemsep1pt\parskip0pt\parsep0pt
\item
  \texttt{obj} Object
\end{itemize}

Free memory that has been allocated to an object via
\texttt{smalloc.alloc}.

\begin{Shaded}
\begin{Highlighting}[]
\KeywordTok{var} \NormalTok{a = \{\};}
\OtherTok{smalloc}\NormalTok{.}\FunctionTok{alloc}\NormalTok{(}\DecValTok{3}\NormalTok{, a);}

\CommentTok{// \{ '0': 0, '1': 0, '2': 0 \}}

\OtherTok{smalloc}\NormalTok{.}\FunctionTok{dispose}\NormalTok{(a);}

\CommentTok{// \{\}}
\end{Highlighting}
\end{Shaded}

This is useful to reduce strain on the garbage collector, but developers
must be careful. Cryptic errors may arise in applications that are
difficult to trace.

\begin{Shaded}
\begin{Highlighting}[]
\KeywordTok{var} \NormalTok{a = }\OtherTok{smalloc}\NormalTok{.}\FunctionTok{alloc}\NormalTok{(}\DecValTok{4}\NormalTok{);}
\KeywordTok{var} \NormalTok{b = }\OtherTok{smalloc}\NormalTok{.}\FunctionTok{alloc}\NormalTok{(}\DecValTok{4}\NormalTok{);}

\CommentTok{// perform this somewhere along the line}
\OtherTok{smalloc}\NormalTok{.}\FunctionTok{dispose}\NormalTok{(b);}

\CommentTok{// now trying to copy some data out}
\OtherTok{smalloc}\NormalTok{.}\FunctionTok{copyOnto}\NormalTok{(b, }\DecValTok{2}\NormalTok{, a, }\DecValTok{0}\NormalTok{, }\DecValTok{2}\NormalTok{);}

\CommentTok{// now results in:}
\CommentTok{// RangeError: copy_length > source_length}
\end{Highlighting}
\end{Shaded}

After \texttt{dispose()} is called object still behaves as one with
external data, for example \texttt{smalloc.hasExternalData()} returns
\texttt{true}. \texttt{dispose()} does not support Buffers, and will
throw if passed.

\subsubsection{smalloc.hasExternalData(obj)}\label{smalloc.hasexternaldataobj}

\begin{itemize}
\itemsep1pt\parskip0pt\parsep0pt
\item
  \texttt{obj} \{Object\}
\end{itemize}

Returns \texttt{true} if the \texttt{obj} has externally allocated
memory.

\subsubsection{smalloc.kMaxLength}\label{smalloc.kmaxlength}

Size of maximum allocation. This is also applicable to Buffer creation.

\subsubsection{smalloc.Types}\label{smalloc.types}

Enum of possible external array types. Contains:

\begin{itemize}
\itemsep1pt\parskip0pt\parsep0pt
\item
  \texttt{Int8}
\item
  \texttt{Uint8}
\item
  \texttt{Int16}
\item
  \texttt{Uint16}
\item
  \texttt{Int32}
\item
  \texttt{Uint32}
\item
  \texttt{Float}
\item
  \texttt{Double}
\item
  \texttt{Uint8Clamped}
\end{itemize}
