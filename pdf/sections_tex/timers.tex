\section{Timers}

\begin{Shaded}
\begin{Highlighting}[]
\DataTypeTok{Stability}\NormalTok{: }\DecValTok{5} \NormalTok{- Locked}
\end{Highlighting}
\end{Shaded}

All of the timer functions are globals. You do not need to
\texttt{require()} this module in order to use them.

\subsection{setTimeout(callback, delay, {[}arg{]}, {[}\ldots{}{]})}

To schedule execution of a one-time \texttt{callback} after
\texttt{delay} milliseconds. Returns a \texttt{timeoutId} for possible
use with \texttt{clearTimeout()}. Optionally you can also pass arguments
to the callback.

It is important to note that your callback will probably not be called
in exactly \texttt{delay} milliseconds - Node.js makes no guarantees
about the exact timing of when the callback will fire, nor of the
ordering things will fire in. The callback will be called as close as
possible to the time specified.

\subsection{clearTimeout(timeoutId)}

Prevents a timeout from triggering.

\subsection{setInterval(callback, delay, {[}arg{]}, {[}\ldots{}{]})}

To schedule the repeated execution of \texttt{callback} every
\texttt{delay} milliseconds. Returns a \texttt{intervalId} for possible
use with \texttt{clearInterval()}. Optionally you can also pass
arguments to the callback.

\subsection{clearInterval(intervalId)}

Stops a interval from triggering.

\subsection{unref()}

The opaque value returned by \texttt{setTimeout} and
\texttt{setInterval} also has the method \texttt{timer.unref()} which
will allow you to create a timer that is active but if it is the only
item left in the event loop won't keep the program running. If the timer
is already \texttt{unref}d calling \texttt{unref} again will have no
effect.

In the case of \texttt{setTimeout} when you \texttt{unref} you create a
separate timer that will wakeup the event loop, creating too many of
these may adversely effect event loop performance -- use wisely.

\subsection{ref()}

If you had previously \texttt{unref()}d a timer you can call
\texttt{ref()} to explicitly request the timer hold the program open. If
the timer is already \texttt{ref}d calling \texttt{ref} again will have
no effect.

\subsection{setImmediate(callback, {[}arg{]}, {[}\ldots{}{]})}

To schedule the ``immediate'' execution of \texttt{callback}. Returns an
\texttt{immediateId} for possible use with \texttt{clearImmediate()}.
Optionally you can also pass arguments to the callback.

Immediates are queued in the order created, and are popped off the queue
once per loop iteration. This is different from
\texttt{process.nextTick} which will execute
\texttt{process.maxTickDepth} queued callbacks per iteration.
\texttt{setImmediate} will yield to the event loop after firing a queued
callback to make sure I/O is not being starved. While order is preserved
for execution, other I/O events may fire between any two scheduled
immediate callbacks.

\subsection{clearImmediate(immediateId)}

Stops an immediate from triggering.
